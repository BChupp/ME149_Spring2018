% This file contains the header data for all assignment files
\documentclass[onecolumn, 11pt]{article}

\usepackage{bbold}   %Get fancy double struck math notation for sets
\usepackage{cite}
\usepackage{cleveref}
\usepackage{color}
\usepackage{courier}   %Have code written out nicely
\usepackage{float}
\usepackage[top=1in, bottom=1in, left=1in, right=1in]{geometry}
\usepackage{graphicx} % handles graphics and figures
\usepackage{hyperref}
\usepackage{listings}
\usepackage{mathtools,bm}
\usepackage{multicol}

\catcode`\^^M=10      %  Makes blank lines meaningless, force use of \par

\definecolor{magenta}{rgb}{0.8,0.0,1.0}
\definecolor{darkGreen}{rgb}{0.0,0.4,0.0}
\definecolor{blue}{rgb}{0.0, 0.0, 0.9}
\definecolor{purple}{rgb}{0.7, 0.0, 0.7}
\definecolor{darkGreen}{rgb}{0.0,0.4,0.0}

\newcommand{\quotes}[1]{``#1''}
\newcommand{\todo}[1]{{\color{magenta}\par {[{\bf ToDo: } {\em #1}} ] \\    }}

\newcommand{\norm}[1]{\left\lVert#1\right\rVert}

%%%%%%%%%%%%%%%%%%%%%%%%%%%%%%%%%%%%%%%%%%%%%%%%%%%%%%%%%%%%%%%%%%%%%%%%%%%%%%%
% NOTE:
%
% The following block of commands is used to format Matlab code blocks for the
% listings package. This block of code is based on two examples:
%  -->   https://gist.github.com/eyliu/120689
%  -->   http://links.tedpavlic.com/ascii/homework_new_tex.ascii
%
% Import file block using:
%   \lstinputlisting{fileName.m}
%
\lstloadlanguages{Matlab}
%
\lstset{language=Matlab,                        % Use MATLAB
        frame=single,                           % Single frame around code
        basicstyle=\small\ttfamily,             % Use small true type font
        keywordstyle=[1]\color{blue}\bfseries,  % MATLAB functions bold and blue
        keywordstyle=[2]\color{purple},         % MATLAB function arguments purple
        keywordstyle=[3]\color{blue}\underbar,  % User functions underlined and blue
        identifierstyle=,                       % Nothing special about identifiers
                                                % Comments small dark green courier
        commentstyle=\usefont{T1}{pcr}{m}{sl}\color{darkGreen}\small,
        stringstyle=\color{darkGreen},            % Strings are purple
        showstringspaces=false,                 % Don't put marks in string spaces
        tabsize=4,                              % 4 spaces per tab
        %
        %%% Put standard MATLAB functions not included in the default
        %%% language here
        morekeywords={xlim,ylim,var,alpha,factorial,poissrnd,normpdf,normcdf},
        %
        %%% Put MATLAB function parameters here
        morekeywords=[2]{on, off, interp},
        %
        %%% Put user defined functions here
        morekeywords=[3]{FindESS, homework_example},
        %
        morecomment=[l][\color{blue}]{...},     % Line continuation (...) like blue comment
        numbers=left,                           % Line numbers on left
        firstnumber=1,                          % Line numbers start with line 1
        numberstyle=\tiny\color{blue},          % Line numbers are blue
        stepnumber=5                            % Line numbers go in steps of 5
        }
%
%%%%%%%%%%%%%%%%%%%%%%%%%%%%%%%%%%%%%%%%%%%%%%%%%%%%%%%%%%%%%%%%%%%%%%%%%%%%%%%


%========================================================================
\title{ME 149:  Final Project  --  [DRAFT]}
\date{Assigned: April 3  ---  Due: April 29 at 11:55pm}
\author{Optimal Control for Robotics}
%========================================================================
\begin{document}
\maketitle

%=================================================

\section*{Introduction}

In the final project for ME 149 you will
implement your own trajectory optimization code
to solve two or more problems of your choosing.
You will start with the basic problem formulations that we have used in the homework
and then add a few features to your code to make the project interesting.

%~~~~~~~~~~~~~~~~~~~~~~~~~~~~~~~~~~~~~~~~~~~~~~~~~~~~~~~~~~~~~~~~~~~~~~~~~~~~~

\section*{Working alone or with a partner}

For the final project you may choose to work alone or with a partner.
If you choose to work with a partner then you will need to propose a more
challenging project (see Advanced Requirements below).
How does grading work with a partner?
You will submit a single proposal and a single final report,
and you will both receive the same grade for the final project.

%~~~~~~~~~~~~~~~~~~~~~~~~~~~~~~~~~~~~~~~~~~~~~~~~~~~~~~~~~~~~~~~~~~~~~~~~~~~~~

\section*{Basic Requirements}

In this project you will implement your own trajectory optimization code.
The basic requirements, listed below, must be met by every project submission.

\vspace{-0.0em} \begin{itemize}  \setlength\itemsep{0em} \setlength\itemindent{18pt}

  \item Implement a direct collocation transcription method

  \item Solve two different trajectory optimization problems

  \item Both optimization problems must be solved using the same transcription function ---
        in other words, your transcription should be somewhat general purpose.

  \item Produce an estimate of the method error that accumulates over each segment
        of the transcription method.

  \item The solution of each optimization problem should be clearly conveyed
        through a set of plots, including initial guess and error estimates.

  \item Your solution should be in the form of an interpolating spline in
        addition to the values at the grid points.

  \item You code should be well written and documented.

\end{itemize}


%~~~~~~~~~~~~~~~~~~~~~~~~~~~~~~~~~~~~~~~~~~~~~~~~~~~~~~~~~~~~~~~~~~~~~~~~~~~~~

\section*{Advanced Requirements}

In addition to the basic requirements, each project must include some advanced
features. I've included a few ideas below, each of which is assigned a
difficulty score of one to four points. Your project proposal must have at
least three points (for one person) or five points (if working in a pair).
You may count the single-point feature multiple times
(for example, you would get one point for
each additional different optimization problem that you solved).
You may suggest additional features in your project proposal if you have ideas
that are not listed below.

\vspace{-0.0em} \begin{itemize}  \setlength\itemsep{0em} \setlength\itemindent{18pt}

\item Constraints not found in homework problems for this course.
      \textbf{[ 1 pt ]}

\item Objective function not found in homework problems for this course.
      \textbf{[ 1 pt ]}

\item Use a dynamical system that is not found in the code library for this course.
      \textbf{[ 1 pt ]}

\item Implement a forward simulation of your solution using ode45 to compute
      an additional error estimate, then compare the two estimates.
      \textbf{[ 1 pt ]}

\item Solve an additional (different) optimization problem.
      \textbf{[ 1 pt ]}

\item Use Hermite--Simpson direct collocation or another medium-order method.
      \textbf{[ 1 pt ]}

\item Implement a tracking controller and forward simulation using the solution
      to one or both of your optimization problems.
      \textbf{[ 2-3 pt ]}

\item Solve a multi-phase trajectory optimization problem for one or both systems.
      \textbf{[2-3 pt]}

\item Implement analytic gradients in your optimization code and use them for
      one or both of your systems. Be sure to verify gradients with FMINCON.
      \textbf{[ 3-4 pt ]}

\item Implement high-order (orthogonal) direct collocation. Points vary by the
      generality of your meshing scheme:
      global collocation / multi-segment, fixed-order / multi-segment, mixed-order.
      \textbf{[ 2-4 pt ]}

\end{itemize}

%~~~~~~~~~~~~~~~~~~~~~~~~~~~~~~~~~~~~~~~~~~~~~~~~~~~~~~~~~~~~~~~~~~~~~~~~~~~~~

\section*{Proposal  --- due April 5}

Create a brief proposal (one side of one page) that outlines your plan for the
final project. Print out your proposal and bring it to class with you on Thursday April 5.

It should include:

\vspace{-0.0em} \begin{itemize}  \setlength\itemsep{0em} \setlength\itemindent{18pt}

\item Are you working alone or in a pair? If in a pair, list both partners names.

\item Which dynamical system (or systems) are you using?

\item Informal description for each optimization problem.
      Clearly identify the objective and constraint functions.

\item Which transcription technique will you be using?

\item Which advanced requirements have you selected?

\end{itemize}

%~~~~~~~~~~~~~~~~~~~~~~~~~~~~~~~~~~~~~~~~~~~~~~~~~~~~~~~~~~~~~~~~~~~~~~~~~~~~~

\section*{Final Project Submission  --- due April 29 at 11:55pm}

\subsection*{Report and appendix}
The primary deliverable for the final project will be a written report,
which should be submitted as a pdf document in the style of a conference paper.
The report should contain a clear description of the project and the methods that were used.
The main body of the report should be not more than four pages with reasonable formatting.
The appendix may be as long as you like, although each section in the appendix should be
referenced in the main body of the report.

\subsection*{Source Code}
The source code for your project should be included as a single compressed archive;
do not include it in the pdf appendix. It should be well documented, including a
README.md file at the top level that describes what each function does and where the
entry point scripts (one for each problem) are. It should also indicate what dependencies
your code has.

\subsection*{Media: video  [optional]}
If your project includes an interesting animation, then you might find it useful to create a video.
You can also use the video to explain your methods or other features of the project
This can either be included in the upload with your report and source code, or posted to YouTube
with a link in your report. Limit the duration of the video to 3 minutes.

%=================================================
\end{document}
