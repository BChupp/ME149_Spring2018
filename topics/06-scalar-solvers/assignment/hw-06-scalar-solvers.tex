% This file contains the header data for all assignment files
\documentclass[onecolumn, 11pt]{article}

\usepackage{bbold}   %Get fancy double struck math notation for sets
\usepackage{cite}
\usepackage{cleveref}
\usepackage{color}
\usepackage{courier}   %Have code written out nicely
\usepackage{float}
\usepackage[top=1in, bottom=1in, left=1in, right=1in]{geometry}
\usepackage{graphicx} % handles graphics and figures
\usepackage{hyperref}
\usepackage{listings}
\usepackage{mathtools,bm}
\usepackage{multicol}

\catcode`\^^M=10      %  Makes blank lines meaningless, force use of \par

\definecolor{magenta}{rgb}{0.8,0.0,1.0}
\definecolor{darkGreen}{rgb}{0.0,0.4,0.0}
\definecolor{blue}{rgb}{0.0, 0.0, 0.9}
\definecolor{purple}{rgb}{0.7, 0.0, 0.7}
\definecolor{darkGreen}{rgb}{0.0,0.4,0.0}

\newcommand{\quotes}[1]{``#1''}
\newcommand{\todo}[1]{{\color{magenta}\par {[{\bf ToDo: } {\em #1}} ] \\    }}

\newcommand{\norm}[1]{\left\lVert#1\right\rVert}

%%%%%%%%%%%%%%%%%%%%%%%%%%%%%%%%%%%%%%%%%%%%%%%%%%%%%%%%%%%%%%%%%%%%%%%%%%%%%%%
% NOTE:
%
% The following block of commands is used to format Matlab code blocks for the
% listings package. This block of code is based on two examples:
%  -->   https://gist.github.com/eyliu/120689
%  -->   http://links.tedpavlic.com/ascii/homework_new_tex.ascii
%
% Import file block using:
%   \lstinputlisting{fileName.m}
%
\lstloadlanguages{Matlab}
%
\lstset{language=Matlab,                        % Use MATLAB
        frame=single,                           % Single frame around code
        basicstyle=\small\ttfamily,             % Use small true type font
        keywordstyle=[1]\color{blue}\bfseries,  % MATLAB functions bold and blue
        keywordstyle=[2]\color{purple},         % MATLAB function arguments purple
        keywordstyle=[3]\color{blue}\underbar,  % User functions underlined and blue
        identifierstyle=,                       % Nothing special about identifiers
                                                % Comments small dark green courier
        commentstyle=\usefont{T1}{pcr}{m}{sl}\color{darkGreen}\small,
        stringstyle=\color{darkGreen},            % Strings are purple
        showstringspaces=false,                 % Don't put marks in string spaces
        tabsize=4,                              % 4 spaces per tab
        %
        %%% Put standard MATLAB functions not included in the default
        %%% language here
        morekeywords={xlim,ylim,var,alpha,factorial,poissrnd,normpdf,normcdf},
        %
        %%% Put MATLAB function parameters here
        morekeywords=[2]{on, off, interp},
        %
        %%% Put user defined functions here
        morekeywords=[3]{FindESS, homework_example},
        %
        morecomment=[l][\color{blue}]{...},     % Line continuation (...) like blue comment
        numbers=left,                           % Line numbers on left
        firstnumber=1,                          % Line numbers start with line 1
        numberstyle=\tiny\color{blue},          % Line numbers are blue
        stepnumber=5                            % Line numbers go in steps of 5
        }
%
%%%%%%%%%%%%%%%%%%%%%%%%%%%%%%%%%%%%%%%%%%%%%%%%%%%%%%%%%%%%%%%%%%%%%%%%%%%%%%%


%========================================================================
\title{Assignment 6:  Scalar Solvers}
\date{Assigned:  Feb 20  ---  Due:  Feb 28 at 11:55pm}
\author{Tufts ME 149:  Optimal Control For Robotics}
%========================================================================
\begin{document}
\maketitle
%=================================================

\section*{Introduction}

In this assignment you will implement two iterative solvers in Matlab:
Ridder's method for scalar root finding and the Golden Section Search for
scalar optimization. Both methods start with a bracketed search interval.

\section*{Ridder's Method}

Ridder's method is an algorithm that searches for the root of a smooth function,
starting with an interval that brackets the root.
The method works by fitting an exponential through three data points on each
iteration and then computing the root of that exponential.
Ridder's method is described in section 9.2 of Numerical Recipes in C.

\section*{Golden Section Search}

The golden section search is method for robust optimization of a
scalar function on a bounded (bracketed) interval.
The algorithm keeps track of three points that are known to bracket the minimum.
On each iteration it samples a new point such that the relative size of the
two sub-intervals remains constant. It turns out that the relative size is
given by the golden ratio, hence the name.
See section 10.1 of Numerical Recipes in C for implementation details.

\section*{Deliverables}

Implement Ridder's method and the golden section search
using the template files that are included with the assignment,
and submit both of these files to Trunk.
You code will be run through an automated unit test,
so make sure that to test it yourself on several functions to be sure that it works.
Carefully read the inputs and output of each function.

\section*{Write-Up}

In addition to the Matlab file for you root solver and your optimization,
please also upload a short write-up for this assignment: \\
\texttt{hw\_06\_studentName\_writeup.txt}
\vspace{-0.0em} \begin{itemize}  \setlength\itemsep{0em} \setlength\itemindent{18pt}
  \item Header: full name, date, assignment name and number
  \item List any other students that you worked with.
  \item List any other references that you used.
  \item How long did this assignment take you?
  \item Briefly describe how you tested each function to verify that it was implemented correctly.
\end{itemize}

\section*{Root Solving Challenge}

You may choose to implement Brent's method for root finding instead of Ridder's method.
Use the template that is provided,
but change the name to \texttt{brentRootSolver} and update the documentation.
You will receive an extra 5 points on your assignment, although the max score will still be 50.
This method is described in section 9.3 of Numerical Recipes in C and it is what
Matlab uses to implement the \texttt{fzero} function.

\section*{Optimization Challenge}

You may choose to implement Brent's method for optimization instead of the golden section search.
Use the template that is provided,
but change the name to \texttt{brentOptimization} and update the documentation.
You will receive an extra 10 points on your assignment, although the max score will still be 50.
This method is described in section 10.2 of Numerical Recipes in C and it is what
Matlab uses to implement the \texttt{fminbnd} function.

\pagebreak
\lstinputlisting{riddersMethod.m}

\pagebreak
\lstinputlisting{goldenSection.m}

%=================================================
\end{document}
