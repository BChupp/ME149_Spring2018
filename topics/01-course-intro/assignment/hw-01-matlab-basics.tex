% This file contains the header data for all assignment files
\documentclass[onecolumn, 11pt]{article}

\usepackage{bbold}   %Get fancy double struck math notation for sets
\usepackage{cite}
\usepackage{cleveref}
\usepackage{color}
\usepackage{courier}   %Have code written out nicely
\usepackage{float}
\usepackage[top=1in, bottom=1in, left=1in, right=1in]{geometry}
\usepackage{graphicx} % handles graphics and figures
\usepackage{hyperref}
\usepackage{listings}
\usepackage{mathtools,bm}
\usepackage{multicol}

\catcode`\^^M=10      %  Makes blank lines meaningless, force use of \par

\definecolor{magenta}{rgb}{0.8,0.0,1.0}
\definecolor{darkGreen}{rgb}{0.0,0.4,0.0}
\definecolor{blue}{rgb}{0.0, 0.0, 0.9}
\definecolor{purple}{rgb}{0.7, 0.0, 0.7}
\definecolor{darkGreen}{rgb}{0.0,0.4,0.0}

\newcommand{\quotes}[1]{``#1''}
\newcommand{\todo}[1]{{\color{magenta}\par {[{\bf ToDo: } {\em #1}} ] \\    }}

\newcommand{\norm}[1]{\left\lVert#1\right\rVert}

%%%%%%%%%%%%%%%%%%%%%%%%%%%%%%%%%%%%%%%%%%%%%%%%%%%%%%%%%%%%%%%%%%%%%%%%%%%%%%%
% NOTE:
%
% The following block of commands is used to format Matlab code blocks for the
% listings package. This block of code is based on two examples:
%  -->   https://gist.github.com/eyliu/120689
%  -->   http://links.tedpavlic.com/ascii/homework_new_tex.ascii
%
% Import file block using:
%   \lstinputlisting{fileName.m}
%
\lstloadlanguages{Matlab}
%
\lstset{language=Matlab,                        % Use MATLAB
        frame=single,                           % Single frame around code
        basicstyle=\small\ttfamily,             % Use small true type font
        keywordstyle=[1]\color{blue}\bfseries,  % MATLAB functions bold and blue
        keywordstyle=[2]\color{purple},         % MATLAB function arguments purple
        keywordstyle=[3]\color{blue}\underbar,  % User functions underlined and blue
        identifierstyle=,                       % Nothing special about identifiers
                                                % Comments small dark green courier
        commentstyle=\usefont{T1}{pcr}{m}{sl}\color{darkGreen}\small,
        stringstyle=\color{darkGreen},            % Strings are purple
        showstringspaces=false,                 % Don't put marks in string spaces
        tabsize=4,                              % 4 spaces per tab
        %
        %%% Put standard MATLAB functions not included in the default
        %%% language here
        morekeywords={xlim,ylim,var,alpha,factorial,poissrnd,normpdf,normcdf},
        %
        %%% Put MATLAB function parameters here
        morekeywords=[2]{on, off, interp},
        %
        %%% Put user defined functions here
        morekeywords=[3]{FindESS, homework_example},
        %
        morecomment=[l][\color{blue}]{...},     % Line continuation (...) like blue comment
        numbers=left,                           % Line numbers on left
        firstnumber=1,                          % Line numbers start with line 1
        numberstyle=\tiny\color{blue},          % Line numbers are blue
        stepnumber=5                            % Line numbers go in steps of 5
        }
%
%%%%%%%%%%%%%%%%%%%%%%%%%%%%%%%%%%%%%%%%%%%%%%%%%%%%%%%%%%%%%%%%%%%%%%%%%%%%%%%


%========================================================================
\title{Assignment 1:  Matlab Basics}
\date{Assigned:  Jan. 18  ---  Due:  Jan. 25}
\author{Optimal Control For Robotics}
%========================================================================
\begin{document} %
\maketitle
%=================================================

\section*{Problem 1: fun with random sequences  (25 pts)}

In this assignment you will write a computer program that generates
a sequence of points in a plane, which will converge to a well-known fractal.
There are two parts to this problem, each with slightly
different rules for generating the sequence of points, resulting in two different fractals.

\par
In addition to submitting plots for each part (on a single figure), you will
also be asked to submit your code and detail how long the assignment took you
to complete. Make sure that your code is well documented (use comments!) and
clearly written so that someone else can make sense of it.

\subsection*{Part One:  a special triangle}
Start by selecting three control points ($A$, $B$, $C$)
that are uniformly spaced around the edge of a circle.\\
Next, generate a sequence of points $P_0 ... P_N$
using a random number generator and the control points. \\
Use $\alpha = \tfrac{1}{2}$ and
the choice of $A$, $B$, or $C$ should be randomly selected with equal probability.

\begin{align}
  P_0 &= A \\
  P_{k+1} &= \alpha \cdot P_k + (1 - \alpha) \cdot \texttt{RandomChoice}(A, B, C)
\end{align}

Create a plot to visualize the sequence, using the following guidelines:
\vspace{-0.3em} \begin{itemize}  \setlength\itemsep{0em}
\item plot the circle as a thin curved black line
\item plot each control point ($A$, $B$, $C$) as a small red circle
\item plot the initial point $P_0$ as a small green ``X''
\item plot $P_1 ... P_N$ as tiny blue dots, where N is 5000
\item make sure that the axes are scaled so that the circle looks like a circle
\end{itemize}

\subsection*{Part Two:  a special rectangle}

Repeat the entire procedure from Part One, but make two changes:
\vspace{-0.6em} \begin{itemize}  \setlength\itemsep{0em}
  \item Select four uniformly spaced control points around the edge of the circle, instead of three.
  \item Set $\alpha = \tfrac{1}{3}$
\end{itemize}

\subsection*{Deliverables:}

\begin{enumerate}
  \item Submit your matlab code as a single file:  \texttt{prob\_01\_studentName.m}
  \item Submit a single figure (with two sub-plots) as a pdf:  \texttt{prob\_01\_studentName.pdf}
  \item Create a short write-up for the problem:   \texttt{prob\_01\_studentName.txt}
\end{enumerate}

\vspace{-1.0em} \begin{itemize}  \setlength\itemsep{0em} \setlength\itemindent{18pt}
  \item Header: full name, studentName, date, problem name and number
  \item List any other students that you worked with.
  \item How long did this problem take you?
  \item Briefly describe or outline your code (roughly 50 -- 100 words).
\end{itemize}

\subsection*{Comments:}
  \vspace{-0.3em} \begin{itemize}  \setlength\itemsep{0em}
    \item In part one you generated the \textit{Sierpinski Triangle} fractal.
          This algorithm is just one of many ways to create this fractal.
    \item This problem was inspired by the numberphile video \quotes{Chaos Game}: \\
          \url{https://www.youtube.com/watch?v=kbKtFN71Lfs}
    \item See the Hints section at the end of the assignment for
          how to save a Matlab figure into a pdf and
          how to include multiple matlab functions in a single file.
\end{itemize}

\section*{Problem 2: simple plots and derivatives (25 pts)}

In this problem you will generate a single figure with a set of six-subplots.
The figure is described below, along with the functions that you will be plotting.
As with the previous problem, please keep track of the time you spend on this problem
and make your code well-documented and organized.
You may use any method that you like to compute the derivatives of $x(t)$ and $y(t)$.

\begin{equation}
x(t) = \big(1 + (t-2)^2 \big) \cdot \sin(3 t)
\end{equation}

\begin{equation}
y(t) = t^3 - 6 t^2 + 2 t + 5
\end{equation}

\begin{equation}
  \dot{z}(t) \equiv \frac{d}{dt} z(t)
  \quad \quad
  \ddot{z}(t) \equiv \frac{d^2}{dt^2} z(t)
\end{equation}

\subsection*{What to plot:}

\vspace{-0.6em} \begin{itemize}  \setlength\itemsep{0em} \setlength\itemindent{18pt}
\item plot $x(t)$, $\dot{x}(t)$, and $\ddot{x}(t)$ on the domain $t \in [0,5]$
\item plot $y(t)$, $\dot{y}(t)$, and $\ddot{y}(t)$ on the domain $t \in [0,5]$
\item the plots for $x(t)$ and its derivatives should be in the left column
\item the plots for $y(t)$ and its derivatives should be in the right column
\item $x(t)$ and $y(t)$ should be in top two plots
\item $\dot{x}(t)$ and $\dot{y}(t)$ should be in middle two plots
\item $\ddot{x}(t)$ and $\ddot{y}(t)$ should be in bottom two plots
\item Plot a blue circle at the point where $y(t)$ reaches its maximum value on the domain $t \in [0,5]$
\item Plot a red ``X'' at the point where $y(t)$ reaches its minimum value on the domain $t \in [0,5]$
\item All plots should have both axes labeled and a title.
\end{itemize}

\subsection*{Deliverables:}

\begin{enumerate}
  \item Submit your matlab code as a single file:  \texttt{prob\_02\_studentName.m}
  \item Create a single figure, with six sub-plots:  \texttt{prob\_02\_studentName.pdf}
  \item Create a short write-up for the problem:   \texttt{prob\_02\_studentName.txt}
\end{enumerate}

\vspace{-1.0em} \begin{itemize}  \setlength\itemsep{0em} \setlength\itemindent{18pt}
  \item Header: full name, studentName, date, problem name and number
  \item List any other students that you worked with.
  \item How long did this problem take you?
  \item Briefly describe your code (roughly 50 -- 100 words).
  \item Show work for derivative calculations (use pseudo-code, \textit{e.g.} $\dot{x} \to $ \texttt{dx}).
  \item How did you compute the minimum and maximum values?
\end{itemize}

\pagebreak

\subsection*{Matlab Hints:}

\vspace{1.0em}

You can export a matlab figure to a .pdf file using the following function:
\lstinputlisting{../../../../codeLibrary/utilities/saveFigureToPdf.m}

\vspace{1.0em}

You can use multiple functions in a single file:
\lstinputlisting{prob_01_studentName.m}

%=================================================
\end{document}
