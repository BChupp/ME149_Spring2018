% This file contains the header data for all assignment files
\documentclass[onecolumn, 11pt]{article}

\usepackage{bbold}   %Get fancy double struck math notation for sets
\usepackage{cite}
\usepackage{cleveref}
\usepackage{color}
\usepackage{courier}   %Have code written out nicely
\usepackage{float}
\usepackage[top=1in, bottom=1in, left=1in, right=1in]{geometry}
\usepackage{graphicx} % handles graphics and figures
\usepackage{hyperref}
\usepackage{listings}
\usepackage{mathtools,bm}
\usepackage{multicol}

\catcode`\^^M=10      %  Makes blank lines meaningless, force use of \par

\definecolor{magenta}{rgb}{0.8,0.0,1.0}
\definecolor{darkGreen}{rgb}{0.0,0.4,0.0}
\definecolor{blue}{rgb}{0.0, 0.0, 0.9}
\definecolor{purple}{rgb}{0.7, 0.0, 0.7}
\definecolor{darkGreen}{rgb}{0.0,0.4,0.0}

\newcommand{\quotes}[1]{``#1''}
\newcommand{\todo}[1]{{\color{magenta}\par {[{\bf ToDo: } {\em #1}} ] \\    }}

\newcommand{\norm}[1]{\left\lVert#1\right\rVert}

%%%%%%%%%%%%%%%%%%%%%%%%%%%%%%%%%%%%%%%%%%%%%%%%%%%%%%%%%%%%%%%%%%%%%%%%%%%%%%%
% NOTE:
%
% The following block of commands is used to format Matlab code blocks for the
% listings package. This block of code is based on two examples:
%  -->   https://gist.github.com/eyliu/120689
%  -->   http://links.tedpavlic.com/ascii/homework_new_tex.ascii
%
% Import file block using:
%   \lstinputlisting{fileName.m}
%
\lstloadlanguages{Matlab}
%
\lstset{language=Matlab,                        % Use MATLAB
        frame=single,                           % Single frame around code
        basicstyle=\small\ttfamily,             % Use small true type font
        keywordstyle=[1]\color{blue}\bfseries,  % MATLAB functions bold and blue
        keywordstyle=[2]\color{purple},         % MATLAB function arguments purple
        keywordstyle=[3]\color{blue}\underbar,  % User functions underlined and blue
        identifierstyle=,                       % Nothing special about identifiers
                                                % Comments small dark green courier
        commentstyle=\usefont{T1}{pcr}{m}{sl}\color{darkGreen}\small,
        stringstyle=\color{darkGreen},            % Strings are purple
        showstringspaces=false,                 % Don't put marks in string spaces
        tabsize=4,                              % 4 spaces per tab
        %
        %%% Put standard MATLAB functions not included in the default
        %%% language here
        morekeywords={xlim,ylim,var,alpha,factorial,poissrnd,normpdf,normcdf},
        %
        %%% Put MATLAB function parameters here
        morekeywords=[2]{on, off, interp},
        %
        %%% Put user defined functions here
        morekeywords=[3]{FindESS, homework_example},
        %
        morecomment=[l][\color{blue}]{...},     % Line continuation (...) like blue comment
        numbers=left,                           % Line numbers on left
        firstnumber=1,                          % Line numbers start with line 1
        numberstyle=\tiny\color{blue},          % Line numbers are blue
        stepnumber=5                            % Line numbers go in steps of 5
        }
%
%%%%%%%%%%%%%%%%%%%%%%%%%%%%%%%%%%%%%%%%%%%%%%%%%%%%%%%%%%%%%%%%%%%%%%%%%%%%%%%


%========================================================================
\title{Assignment 4:  Introduction to Tracking Controllers}
\date{Assigned:  Feb 6  ---  Due:  Feb 15}
\author{Tufts ME 149:  Optimal Control For Robotics}
%========================================================================
\begin{document}
\maketitle
%=================================================

\section*{Introduction}

\todo{Finish this draft! }
\todo{The requirements for the assignment will remain unchanged, but I will
      add more clarification around the details. Final version will be posted
      before Thursday Feb 8. Please email me with any questions or comments
      regarding the assignment. }


\section*{Part One: Cubic Spline Math}

Cubic splines are commonly used to represent trajectories.
Cubic splines are useful because they are easy to work with and
can be constructed with a continuous first derivative (unlike linear splines).
Rather than construct an entire spline, we will focus on a single cubic segment $x(t)$, shown below.

\begin{equation}
x(t) = \sum_{i=0}^3 c_i t^i = c_0 + c_1 t + c_2 t^2 + c_3 t ^ 3
\end{equation}

A cubic Hermite spline is a cubic spline that is fully defined by the value and slope
at each knot point. These boundary conditions are given below.

\begin{align}
x(0) &= x_0
  \quad \quad & \quad \quad
x(h) &= x_h
  \\
\dot{x}(0) &= v_0
  \quad \quad & \quad \quad
\dot{x}(h) &= v_h
\end{align}

\textbf{1a:} Given the boundary conditions and spline equation above,
compute the spline coefficients $c_i$ in terms of the boundary conditions. \\

\textbf{1b:} Cubic splines are often constructed with a constraint that the
acceleration at each boundary is continuous. Compute the acceleration at each
boundary: $\ddot{x}(0)$ and $\ddot{x}(h)$. You may write acceleration in terms
of either the boundary conditions or the spline coefficients. \\

\textbf{Note:} You may perform these calculations either by hand or using the
Matlab symbolic toolbox. If working by hand, then please scan your calculations
(this can typically be done using a simple app on your phone)
and then upload them as a pdf.
If using the Matlab symbolic toolbox, them upload your derivation script.
The script should print the solution to the command prompt; please copy this
output directly into the comments of the script.

\section*{Part Two: Pendulum Swing-Up}

Construct a piecewise-polynomial reference trajectory for a simple pendulum.
The trajectory should be in the form of three pp splines:
\texttt{qRef} (angle),
\texttt{dqRef} (rate), and
\texttt{ddqRef} (accel).
The trajectory should move the system from the
stable equilibrium (\texttt{qRef(0) = 0}) to the
unstable equilibrium configuration (\texttt{qRef(T) = pi}).
The initial and final rates should be zero.
This ``swing-up'' trajectory should take 3 seconds to complete.

\par
You may use any method that you like to construct the pp-spline,
provided that it can be evaluated using the \texttt{ppval()} function in Matlab.
A few possible options include:
\texttt{spline}, \texttt{pchip}, \texttt{pwch},
and \texttt{ME149/codeLibrary/splines/pwqh}.

\par
Make sure that your splines have \textit{consistent} derivatives: the
the rate reference should be the derivative of the angle reference, and the
acceleration reference should be the derivative of the rate reference.
One way to achieve this is by constructing \texttt{qRef} first and then using
the \texttt{ME149/codeLibrary/splines/ppDer} function in the code library
to compute the derivatives for you.

\par
Design a proportional-derivative (PD) controller to track the set of
reference trajectories. You may use any method that you like to tune the gains.
The function\\
\texttt{ME149/codeLibrary/control/secondOrderSystemGains}
in the
code library might be useful, although you are not required to use it.
In addition to the feed-back controller, compute the feed-forward reference
torque using the inverse dynamics for the pendulum.

\par
Use the simple pendulum dynamics model
(\texttt{ME149/codeLibrary/simplePendulum/}) from the code library,
with \texttt{param.freq = 1.0} and \texttt{param.damp = 0.0}.
The library provides functions for both forward and inverse dynamics.

\par
It is often desirable to minimize the ``actuator effort'' along a trajectory.
There are many models for actuator effort, but here we will use the torque-squared model.
Note that the objective function $J()$ is a
\textit{functional}: it is a function that takes another function as input.

\begin{equation}
J\big(u(t)\big) = \int_0^T \! g(\tau) \, d\tau = \int_0^T \! u^2(\tau) \, d\tau
\end{equation}

This type of objective function is a path-integral.
When doing trajectory optimization, the correct way to evaluate a path integral is to
solve it using the same method that you use for propagating the dynamics.
This ensures that your results are self consistent, which will be important later.
To do this, you construct a modified dynamics function at the start of your simulation,
where the integrand $g()$ and the system dynamics $f()$ are stacked to form a single
combined dynamics function that is passed to the simulator.

\par
\textbf{2a:} Compute the torque-squared objective function for your reference trajectory.
This should be done by calling your simulation function with the \texttt{dynFun} = $g()$.

\textbf{2b:} Compute the torque-squared objective function for the simulation of the
full closed-loop system using the controller (with both feed-back and feed-forward terms).
Introduce an intentional perturbation to the initial conditions:
the initial angle for the simulation should be 0.2 radians.
This will allow the simulation to test the performance of your controller.

\todo{Requirements for plots and other output for submission.}

\section*{Part Three: Double Pendulum Swing-Up}

Same as the single pendulum swing-up, but now apply those techniques to a
more complicated system. The initial conditions now start with both links of the
pendulum in stable equilibrium (minimum potential energy) and end with both links
in the unstable equilibrium (maximum potential energy). The path integral can be
applied to both torque (which are added), and you can simply write a PD controller
for each joint. As with the previous assignment, set all parameters in the double
pendulum dynamics (link lengths, link masses, and gravity) to one.

\section*{Implementation Details}

You should be able to create one (or more) functions that work for both the single and double pendulum.
For example, the objective function should work for any number of torque inputs.


%=================================================
\end{document}
