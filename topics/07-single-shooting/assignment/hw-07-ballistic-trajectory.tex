% This file contains the header data for all assignment files
\documentclass[onecolumn, 11pt]{article}

\usepackage{bbold}   %Get fancy double struck math notation for sets
\usepackage{cite}
\usepackage{cleveref}
\usepackage{color}
\usepackage{courier}   %Have code written out nicely
\usepackage{float}
\usepackage[top=1in, bottom=1in, left=1in, right=1in]{geometry}
\usepackage{graphicx} % handles graphics and figures
\usepackage{hyperref}
\usepackage{listings}
\usepackage{mathtools,bm}
\usepackage{multicol}

\catcode`\^^M=10      %  Makes blank lines meaningless, force use of \par

\definecolor{magenta}{rgb}{0.8,0.0,1.0}
\definecolor{darkGreen}{rgb}{0.0,0.4,0.0}
\definecolor{blue}{rgb}{0.0, 0.0, 0.9}
\definecolor{purple}{rgb}{0.7, 0.0, 0.7}
\definecolor{darkGreen}{rgb}{0.0,0.4,0.0}

\newcommand{\quotes}[1]{``#1''}
\newcommand{\todo}[1]{{\color{magenta}\par {[{\bf ToDo: } {\em #1}} ] \\    }}

\newcommand{\norm}[1]{\left\lVert#1\right\rVert}

%%%%%%%%%%%%%%%%%%%%%%%%%%%%%%%%%%%%%%%%%%%%%%%%%%%%%%%%%%%%%%%%%%%%%%%%%%%%%%%
% NOTE:
%
% The following block of commands is used to format Matlab code blocks for the
% listings package. This block of code is based on two examples:
%  -->   https://gist.github.com/eyliu/120689
%  -->   http://links.tedpavlic.com/ascii/homework_new_tex.ascii
%
% Import file block using:
%   \lstinputlisting{fileName.m}
%
\lstloadlanguages{Matlab}
%
\lstset{language=Matlab,                        % Use MATLAB
        frame=single,                           % Single frame around code
        basicstyle=\small\ttfamily,             % Use small true type font
        keywordstyle=[1]\color{blue}\bfseries,  % MATLAB functions bold and blue
        keywordstyle=[2]\color{purple},         % MATLAB function arguments purple
        keywordstyle=[3]\color{blue}\underbar,  % User functions underlined and blue
        identifierstyle=,                       % Nothing special about identifiers
                                                % Comments small dark green courier
        commentstyle=\usefont{T1}{pcr}{m}{sl}\color{darkGreen}\small,
        stringstyle=\color{darkGreen},            % Strings are purple
        showstringspaces=false,                 % Don't put marks in string spaces
        tabsize=4,                              % 4 spaces per tab
        %
        %%% Put standard MATLAB functions not included in the default
        %%% language here
        morekeywords={xlim,ylim,var,alpha,factorial,poissrnd,normpdf,normcdf},
        %
        %%% Put MATLAB function parameters here
        morekeywords=[2]{on, off, interp},
        %
        %%% Put user defined functions here
        morekeywords=[3]{FindESS, homework_example},
        %
        morecomment=[l][\color{blue}]{...},     % Line continuation (...) like blue comment
        numbers=left,                           % Line numbers on left
        firstnumber=1,                          % Line numbers start with line 1
        numberstyle=\tiny\color{blue},          % Line numbers are blue
        stepnumber=5                            % Line numbers go in steps of 5
        }
%
%%%%%%%%%%%%%%%%%%%%%%%%%%%%%%%%%%%%%%%%%%%%%%%%%%%%%%%%%%%%%%%%%%%%%%%%%%%%%%%


%========================================================================
\title{Assignment 07:  Ballistic Trajectory Calculation}
\date{Assigned:  February 27  ---  Due:  March 7 at 11:55pm}
\author{Optimal Control for Robotics}
%========================================================================
\begin{document}
\maketitle
%=================================================

\section*{Introduction}

In this assignment you will solve a simple boundary value problem using single shooting.

The example problem will be that of aiming a canon:
compute the launch velocity vector with the minimum launch speed
such that the projectile reaches the specified target.

\section*{Deliverables}

Implement the function \texttt{computeBallisticTrajectory} using the template provided.
Also submit any additional Matlab code that you write for the assignment.

\section*{Write-Up:}

In addition to the Matlab code described above,
upload a short write-up for this assignment: \\
\texttt{hw\_07\_studentName\_writeup.txt}
\vspace{-0.0em} \begin{itemize}  \setlength\itemsep{0em} \setlength\itemindent{18pt}
  \item Header: full name, date, assignment name and number
  \item List any other students that you worked with.
  \item How long did this assignment take you?
  \item Briefly describe your approach to testing your code.
  \item Answers to the questions below.
\end{itemize}

One of the best ways to learn about trajectory optimization is to experiment with
the inputs and outputs of an optimization.
Here are a few things to investigate.
\vspace{-0.0em} \begin{itemize}  \setlength\itemsep{0em} \setlength\itemindent{18pt}
\item How does varying the number of grid points and simulation method affect the
      number of iterations, solve time, and final accuracy of the NLP solver?
\item Adjust the various problem parameters to see how the solution changes.
      What happens if the drag is large and the distance is large? Why?
\item How many function calls does \texttt{fmincon} make on each iteration?
      What is going on here?
\end{itemize}

\pagebreak
\section*{\texttt{computeBallisticTrajectory.m}}

For this assignment you will implement one Matlab function: \texttt{computeBallisticTrajectory}.
This function takes a set of paramters that describe a specific boundary value problem
and the computes the optimal ballistic trajectory.
In general there are many ways to solve this problem.
For this assignment I will restrict a few of the details to keep things consistent across the class.
\vspace{-0.6em} \begin{itemize}  \setlength\itemsep{0em}
  \item Use the Matlab function \texttt{fmincon} to solve the underlying non-linear program.
        Create a \texttt{problem} data structure to define the NLP and
        then pass it as a single argument to \texttt{fmincon}.
  \item Use the function \texttt{runSimulation} from the code library to run the simulation.
  \item Use a uniform time grid.
  \item Use all of the inputs and return all of the outputs that are specified in the template code.
  \item The decision variables in the optiization should be the initial velocity vector (three elements)
        and the total duration of the simulation.
  \item The objective function should be the initial speed squared.
\end{itemize}


\subsection*{Inputs:}

There are four inputs to this function, which are described in detail in the source code.
The first input, \texttt{param}, describes the boundary value problem specification:
boundary positions and physics model parameters.
The function \texttt{getBallisticParameters} in the assignment source code can be used
to provide a candidate set of parameters.
The second input, \texttt{nGrid}, gives the number of gridpoints
that should be used for the underlying simulation.
Next is \texttt{method} which is passed directly to \texttt{runSimulation}.
The final input, \texttt{nlpOpt}, is a set of options that are passed to \texttt{fmincon}.
See the template code for how to get the default parameters.

\subsection*{Outputs:}

There are two outputs, which describe the optimal trajectory that the ballistic projectile takes.
The first should be the uniform time grid that was used by the simulation for the dynamics.
The second should be the full state (position and velocity) at each point in time.

\pagebreak
\subsection*{Template Code:}

\lstinputlisting{computeBallisticTrajectory.m}


\pagebreak

\section*{Dynamics Model}

We will use a simple model of projectile motion:
the projectile is acted on by only gravity and a quadratic drag force.
The dynamics are implemented in the code library: \\
\texttt{ME149/codeLibrary/modelSystems/projectile/projectileDynamics.m} \\
The equations of motion are given below for the curious.
The constant parameters are:
$d$ = coefficient of quadratic drag,
$g$ = acceleration due to gravity,
$m$ = projectile mass,
$\bm{\vec{w}}$ = wind velocity.
There are three time-varying scalar functions:
$x = x(t)$, $y = y(t)$, and $z = z(t)$ that describe the position of the projectile.
Finally, there are three constant unit vectors:
$ \hat{\bm{i}}$, $\bm{\hat{j}}$, $\hat{\bm{k}}$.

\begin{equation}
  \bm{\vec{p}} = x  \,  \bm{\hat{i}} + y  \,  \bm{\hat{j}} + z  \,  \bm{\hat{k}}
\end{equation}
\vspace{-0.8em}
\begin{equation}
  \bm{\vec{v}} = \bm{\dot{\vec{p}}} - \bm{\vec{w}}
\end{equation}
\vspace{-0.5em}
\begin{equation}
  \bm{\vec{F}_D} = -d   \;  || \bm{\vec{v}} ||  \;  \bm{\vec{v}}
\end{equation}
\vspace{-0.5em}
\begin{equation}
  \bm{\vec{F}_G} = -m  \;  g  \;  \bm{\hat{k}}
\end{equation}
\vspace{-0.5em}
\begin{equation}
  \bm{\ddot{\vec{p}}} = ( \bm{\vec{F}_D} + \bm{\vec{F}_G} ) / m
\end{equation}


%=================================================
\end{document}
